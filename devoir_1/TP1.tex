\documentclass[12pt]{article}
\usepackage[T1]{fontenc}
\usepackage[utf8]{inputenc}
\usepackage[francais]{babel}
\usepackage{amsmath, amssymb}

\newcommand\ra\rightarrow
\newcommand\la\leftarrow
\newcommand\lra\leftrightarrow
\newcommand\ua\uparrow
\newcommand\da\downarrow
\newcommand\uda\updownarrow
\newcommand\nea\nearrow

\newcommand\Ra\Rightarrow
\newcommand\La\Leftarrow
\newcommand\Lra\Leftrightarrow

\newcommand\mat[1]{\begin{bmatrix} #1 \end{bmatrix}}
\newcommand\smat[1]{\setstretch{1}{\ensuremath{\scalefont{0.8}\mat{#1}}}}

\newcommand{\bra}[1]{\ensuremath{\left\langle#1\right|}}
\newcommand{\ket}[1]{\ensuremath{\left|#1\right\rangle}}
\newcommand{\bracket}[2]{\ensuremath{\left\langle #1 \middle| #2 \right\rangle}}
\newcommand{\matrixel}[3]{\ensuremath{\left\langle #1 \middle| #2 \middle| #3 \right\rangle}}
\newenvironment{eq*}{\begin{equation*}\begin{gathered}}{\end{gathered}\end{equation*}}
\newenvironment{eqs*}{\begin{equation*}\begin{aligned}}{\end{aligned}\end{equation*}}

% Paragraph formatting
\setlength{\parindent}{0pt}
\usepackage{setspace}
\onehalfspacing
% Margin
\usepackage[margin=1in]{geometry}
\setcounter{secnumdepth}{0}

% Doc info
\author{Olivier St-Laurent, Maxime Daigle}
\title{IFT-3395  Devoir 1}
\date{2018-09-27}

\begin{document}

\maketitle

\section{Question 1}

Nous pensons que la conclusion des médecins est fausse. En réalité, puisque la probabilité à priori d'avoir le cancer est très faible (1.5\%), la probabilité de ne pas être atteint du cancer, en sachant que le test est positif, est largement supérieure à la probabilité d'être atteinte.
\\[\baselineskip]
Si nous calculons la probabilité exacte avec la loi de Bayes, nous avons:
\begin{eqs*}
P(cancer | +) = \tfrac{P(+ | cancer) * P(cancer)}{P(+ | cancer) * P(cancer) + P(+ | cancer^c) * P(cancer^c)}
\end{eqs*}
\\[\baselineskip]
En remplaçant dans l'équation les valeurs que nous avons, nous obtenons:
\begin{eqs*}
P(cancer | +) = \tfrac{(0.87)(0.015)}{(0.87)(0.015) + (0.096)(1 - 0.015)} = 0.12\%
\end{eqs*}

\section{Question 2}
\subsection{2.1}
Si pour 2 dimensions, l'aire d'un carré de côté $c$ est donné par $c^2$ et que pour 3 dimensions, le volume d'un cube de côté $c$ est donné par $c^3$, alors
on peut généralisé que le volume d'un hyper-cube de coté $c$ en dimension $d$ est donné par $V = c^d$


\subsection{2.2}

Pour une variable aléatoire $X$ uniformément distribuer sur un interval [$\alpha$, $\beta$] de dimension 1, sa fonction de densité de probabilité est donnée par:
\[
    f(x) =
	\begin{cases}
        \tfrac{1}{\beta -\alpha}, & \text{si $\alpha \leq x \leq \beta$} \\
        0, & \text{ sinon}
	\end{cases}
\]
Pour une variable aléatoire vectorielle $X$ de 2 dimensions uniformément distribuée sur une région $R$, elle aussi de deux dimension, la fonction
de densité conjointe est une constante $c$ pour tout point dans $R$ et 0 pour tout point a l'extérieur de $R$
\[
    f(x_{1}, x_{2}) =
	\begin{cases}
        c, & \text{si $(x_{1}, x_{2}) \in R$}  \\
        0, & \text{ sinon}
	\end{cases}
\]

Si on généralise le cas pour $n$ dimensions, on obtient la fonction de densité conjointe:
\[
    f(x_{1}, ...,x_{d}) =
	\begin{cases}
        c, & \text{si $(x_{1}, ..., x_{d}) \in R^d$}  \\
        0, & \text{ sinon}
	\end{cases}
\]

La probabilité $p(x)$ pour une valeur exacte dans un problème de probabilité continue est de 0, car l'intervale est infiniment petit.
\\[\baselineskip]
En effet, la probabilité $p(x)$ dans $d$ dimension peut être calculé par:

\begin{eqs*}
p(x) = \int_{\alpha_{1}}^{\beta_{1}} ... \int_{\alpha_{d}}^{\beta_{d}} f(x_{1}, ...,x_{d})dx_{1}... dx_{d} = c*\text{Volume de }R^d
\end{eqs*}

\subsection{2.3}
Si la  bordure est de 3\%, la probabilité que $x_{i}$, $i \in 1, ..., d$ se retrouve dans cette zone est de $1 - 0.94 = 0.06$ Donc pour un $x$ de l'hyper-cube, la
probabilité de tomber dans le cube et pas dans la bordure est donnée par l'équation:
\begin{eqs*}
p(x) = \int_{0}^{0.94} ... \int_{0}^{0.94}dx_{1}... dx_{d} = \tfrac{0.94^d}{1}
\end{eqs*}
et la probabilité de tomber dans la bordure est de:
\begin{eqs*}
1 - p(x) =1 - \tfrac{0.94^d}{1}
\end{eqs*}

\subsection{2.4}
Pour $d$ = 1
\begin{eqs*}
1 - p(x) =1 - \tfrac{0.94^1}{1} = 0.06
\end{eqs*}

Pour $d$ = 2
\begin{eqs*}
1 - p(x) =1 - \tfrac{0.94^2}{1} = 0.1164
\end{eqs*}

Pour $d$ = 3
\begin{eqs*}
1 - p(x) =1 - \tfrac{0.94^5}{1} = 0.1694
\end{eqs*}

Pour $d$ = 5
\begin{eqs*}
1 - p(x) =1 - \tfrac{0.94^5}{1} = 0.2661
\end{eqs*}

Pour $d$ = 10
\begin{eqs*}
1 - p(x) =1 - \tfrac{0.94^{10}}{1} = 0.4613
\end{eqs*}

Pour $d$ = 100
\begin{eqs*}
1 - p(x) =1 - \tfrac{0.94^{100}}{1} = 0.9979
\end{eqs*}

Pour $d$ = 1000
\begin{eqs*}
1 - p(x) =1 - \tfrac{0.94^{1000}}{1} = 1 - 1.3423*10^{-27}
\end{eqs*}

\newpage
\subsection{2.5}

Plus la dimensionalité est grande, plus la probabilité de tomber dans la bordure augmente. Si on y pense, c'est logique, car lorsque la dimensionalité est de 
100, par exemple, pour tomber dans la bordure, il suffit qu'une seule des 100 dimensions soit dans celle-ci pour ne plus être dans l'hyper-cube. 

\section{Question 3}
\subsection{3.1}
\subsubsection{a)}
\end{document}